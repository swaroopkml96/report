\documentclass[11pt,twoside,a4paper]{report}
\title{Detecting abnormalities on chest X-rays using deep neural networks}
\author{Swaroop Kumar M L}
\begin{document}
\maketitle
\begin{abstract}
    Diseases like pneumonia and tuberculosis are leading causes of death world-wide,
    Although conclusive diagnosis requires other tests such as a sputum culture,
    chest radiography can be an important diagnostic aid and is routinely recommended
    since it is fast, affordable and highly sensitive. Moreover, automated detection of
    abnormalities on the chest X-ray can help in screening and severity-based prioritization.
    There has been increasing interest in using deep learning for computer-aided diagnosis in both the
    machine learning and radiology communities. There has also been recent work in model explainability and
    weakly supervised localization, and methods for dealing with label noise in a weakly supervised setting\\

    Inspired by previous work, we develop algorithms that
    can detect abnormalities on the X-ray and explain these detections using weakly supervised localization.
    We establish baselines, benchmark against previous work and evaluate the effects of recent techniques
    such as mixup on performance and generalizability to other datasets.
    In terms of AUROC, we acheive performance competitive with previous work in a) detecting pneumonia-like
    and other abnormalities on the NIH chestX-ray14 dataset and b) in detecting tuberculosis on the Shenzhen
    hospital dataset, and we achieve state-of-the-art performance on the Montgomery county tuberculosis dataset.
    We also test our algorithms for bias with respect to gender, age and view position. 
\end{abstract}
\tableofcontents
\chapter{Introduction}
    \section{Problem definition}
    \section{Motivation}
    \section{Previous work}
    \section{Our work}
    \section{Report layout}
\chapter{Literature survey}
    \section{CheXNet and CheXNext}
    \section{Weakly supervised learning}
    \section{Explainability}
    \section{Fairness}
    \section{Learning at multiple scales}
    \section{Attention}
    \section{Generalizability}
    \section{Other methods}
    \section{Tuberculosis}
\chapter{Data}
    \section{NIH CXR-14}
        \subsection{Challenges and issues}
    \section{Mendeley}
    \section{Shenzhen and Montgomery}
\chapter{Baselines}
    \section{Model architecture}
    \section{Training procedure}
    \section{Unsupervised localization}
        \subsection{LIME as an alternative}
\chapter{Experiments}
    \section{Data augmentation}
    \section{Test-time augmentation}
    \section{Higher resolution}
    \section{Progressive resizing}
    \section{Ensembling for scale-invariance}
    \section{Location-sensitive ensembling}
    \section{Mixup}
    \section{Label drift}
    Choose a better name
    \section{Transfer learning for tuberculosis}
    \section{Generalizability}
\chapter{Results}
    \section{Comparison metrics}
    \section{Comparison to previous work}
    \section{Comparison to human radiologists}
        \subsection{Caveats with comparison to human radiologists}
    \section{Other important factors}
    Such as how well predicted probabilities correspond to actual severity, localization, and time
    \section{Examples}
\chapter{Bias}
    \section{Gender}
    \section{Age}
    \section{View position}
\chapter{Conclusion}
\chapter{Future work}
    \section{Limitations of proposed work}
    \section{Avenues for further research}
    \section{Practical implementation and clinical relevance}
    Scalability
    \section{More recent datasets}
\end{document}